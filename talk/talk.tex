%\documentclass[handout]{beamer}
\documentclass{beamer}

\input{talkpreamble}

\usepackage{algorithm}
%\usepackage{algorithmic} 
\usepackage[noend]{algpseudocode}

\title[Understanding Differences Between Modelling Pipelines]{Towards Understanding Differences Between Modelling Pipelines: a Modelers Perspective}
\author[R Hoffmann]{Csobán Balogh, \textbf{Ruth Hoffmann} and Joan Espasa}
\institute[ModRef 2024]{ModRef 2024, University of Girona}

\begin{document}

{
    \begin{frame}[plain,noframenumbering]
        \titlepage
    \end{frame}
}


\section*{Intro}
\begin{frame}{What and why?}
From a (single) users perspective:
    \begin{itemize}
        \item Investigate the capabilities of constraint programming pipelines.
        \item Investigate the difference in languages.
        \item Comparing MiniZinc and Savile Row.
    \end{itemize}
\end{frame}

\begin{frame}{How?}
\begin{itemize}
    \item Create as close to equivalent models as possible.
    \item Using MiniZinc and Essence' (not Essence)
    \item Compare over the same solver (Chuffed).
    \item Compare over equivalent(ish) optimisation levels.
    \item 6 Models from different problem classes
    \begin{enumerate}
        \item Quasigroup Completion, (no exciting differences)
        \item Wordpress Problem, (no exciting differences)
        \item Rotating Rostering Problem, (no exciting differences)
        \item Travelling Tournament Problem with Predefined Values, 
        \item Multi-Skilled Project Scheduling Problem,
        \item Capacitated Vehicle Routing Problem with Time Windows.
    \end{enumerate}
\end{itemize}
\end{frame}

\section*{Language Differences}
\begin{frame}{\texttt{regular} (MZn) vs \texttt{forAll} (SR)}
Traveling Tournament Problem with Predefined Venues at most three consecutive away or home games
\begin{itemize}
    \item at most three consecutive away or home games
    \item[MZn] \texttt{regular} asserts that a sequence of variables take a value from a finite automaton.
    \item[E'] \texttt{forAll} checking that there are not four consecutive assignments.
\end{itemize}
\end{frame}

\begin{frame}{\texttt{circuit} (MZn)}
Capacitated Vehicle Routing problem with Time Windows, Service Times and Pickup and Deliveries.
\begin{itemize}
    \item \texttt{circuit} is used to ensure the vehicle delivery routes do not take sub-tours in their route and visits each location uniquely for optimisation
    \item[MnZ] A \texttt{circuit} is such that the cell value of an array points to the index of the next number, and this forms a circuit that continues around. 
    \item[E'] \url{https://github.com/MiniZinc/libminizinc/blob/master/share/minizinc/std/fzn_circuit.mzn}
\end{itemize}
\end{frame}


\begin{frame}{Set Variables}
    Multi-Skilled Project Scheduling Problem
    \begin{itemize}
        \item Sets of skills, workers etc. (each assigned an integer)
        \item[MnZ] Variables which are a set.
        \item[E'] Occurrence representation of the integers/elements.
    \end{itemize} 
\end{frame}

\begin{frame}[containsverbatim]{\texttt{letting} (MZn)}
Multi-Skilled Project Scheduling Problem
\begin{itemize}
    \item \texttt{letting} creates variables within constraints
\end{itemize}
\begin{lstlisting}[basicstyle=\tiny, language=minizinc]
let { set of int: WTasks =
        { i | i in Tasks where  exists(k in has_skills[j])(rr[k, i] > 0) }
} in...
let { set of int: TWorkers = 
        { j | j in Workers where  exists(k in has_skills[j])(rr[k, i] > 0) }
} in...
\end{lstlisting}
\begin{lstlisting}[basicstyle=\tiny, language=eprime]
forAll i : Tasks . forAll j : Workers .
    TWorkers[j, i] = 1 <-> 
        exists k : Skills . has_skills[j, k] = 1 /\ rr[k,i] > 0,
\end{lstlisting}
\end{frame}


\begin{frame}{\texttt{cumulative} (MnZ)}
Multi-Skilled Project Scheduling Problem
\begin{itemize}
    \item Determines whether set of tasks with start times, durations, and resource requirements, never exceed the global resource bound at any time.
    \item[MnZ] Determines if a cumulative resource usage is within bounds.
    \item[E'] \url{https://github.com/MiniZinc/libminizinc/blob/master/share/minizinc/std/fzn_cumulative.mzn}
\end{itemize}
\end{frame}

\section*{Results}
\begin{frame}{Results}
\includegraphics{../python/graphs/paper_total_cactus.png}
\end{frame}

\begin{frame}{Results}
{\footnotesize{

\begin{table}[h!]
    \centering
    \begin{tabular}{c|ccc}
    & \multicolumn{3}{c}{Timing Ratio} \\
    Problem & $\frac{\text{O0S0}}{\text{O0}}$ & $\frac{\text{O2S1}}{\text{O1}}$ & $\frac{\text{O3S2}}{\text{O5}}$ \\ \hline
    Quasigroup           & 0.08      & 0.5       & 0.6   \\        
    Quasigroup Occ.      & 0.12      & 0.08      & 0.38  \\
    Wordpress            & 1.54      & 1.83      & 5.29  \\               
    Wordpress Symm.      & 1.47      & 1.36      & 0.49  \\ 
    TTPPV                & 0.99      & 1.35      & 1.98  \\ 
    MSPSP                & 138.48    & 88.57     & 612.61\\ 
    CVRPTW               & 1.0       & 1.0       & 1.0   \\ 
    Rostering            & 29.5      & 15.15     & 77.7   
    \end{tabular}
\end{table}}}
\end{frame}
    
\section*{Take Away}
\begin{frame}{Take Away}
\begin{itemize}
    \item[MnZ] allows better (expert) modeler control
    \item[MnZ] provides a slightly more expressive language due to the facilities for code organization and reusability
    \item[SR] provides a solid set of default settings
    \item[SR] has a more consistent performance profile
\end{itemize}    

\end{frame}

\section*{}
\begin{frame}[plain,noframenumbering,b]
    \begin{tikzpicture}[remember picture, overlay]
        \node at (current page.north west) {
            \begin{tikzpicture}[remember picture, overlay]
                \fill [fill=uofstablue, anchor=north west] (0, 0) rectangle (\paperwidth, -1.7cm);
            \end{tikzpicture}
        };

        \node (logo) [anchor=north east, shift={(-0.1cm,0.2cm)}] at (current page.north east) {
            \includegraphics[keepaspectratio=true,scale=0.12]{UniStALogos/03-standard-horizontal/02-standard-white.png}
        };
    \end{tikzpicture}

    \begin{center}
        {\Large{Thank you!}} \\ \vspace{1cm}
        \href{https://github.com/ruthhoffmann}{ \faGithub\ ruthhoffmann} \\ [2mm]
        \href{https://www.st-andrews.ac.uk/computer-science/people/rh347/}{ \faDesktop\ www.st-andrews.ac.uk/computer-science/people/rh347/} \\ [2mm]
        \href{mailto:rh347@st-andrews.ac.uk}{\faEnvelope\ \nolinkurl{rh347@st-andrews.ac.uk}} \\ [1cm]
        %Find me in my office \\ [1cm]
    \end{center}
\end{frame}

\end{document}
